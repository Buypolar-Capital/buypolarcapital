\documentclass[12pt]{article}
\usepackage{amsmath, amssymb, graphicx, hyperref}
\usepackage[utf8]{inputenc}
\usepackage[T1]{fontenc}

\title{Sample LaTeX Document}
\author{Your Name}
\date{April 2025}

\begin{document}

\maketitle

\begin{abstract}
This is a sample LaTeX document created in Visual Studio Code to demonstrate basic formatting, mathematics, and references.
\end{abstract} 

\section{Introduction}
Welcome to LaTeX in VS Code! This document showcases a simple structure similar to what you might create in Overleaf. LaTeX is excellent for typesetting scientific documents with complex mathematics, such as the equation below:
\begin{equation} 
E = mc^2
\end{equation}

\section{Mathematics Example}
Here’s an example of a mathematical derivation using the \texttt{amsmath} package:
\begin{align}
f(x) &= \sin(x) \cos(x) \\
     &= \frac{1}{2} \sin(2x) \label{eq:trig}
\end{align}
Equation \eqref{eq:trig} uses a trigonometric identity. You can compile this document with the LaTeX Workshop extension and view the PDF output.

\section{Figures and References}
You can include images using the \texttt{graphicx} package. For example:
\begin{figure}[h]
    \centering
    % \includegraphics[width=0.5\textwidth]{example-image}
    \caption{Placeholder for an image.}
    \label{fig:example}
\end{figure}

For citations, use a \texttt{.bib} file and BibTeX. See \cite{knuth1997art} for an example.

\section{Conclusion}
This document demonstrates a basic LaTeX setup in VS Code. You can extend it with tables, algorithms, or custom packages as needed.

\bibliographystyle{plain}
\bibliography{references}


\section{References}

Hello there boss



\end{document}